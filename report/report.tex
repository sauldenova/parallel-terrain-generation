\documentclass[10pt,letterpaper,oneside]{article}
\usepackage[margin=2.5cm]{geometry}
\usepackage[utf8]{inputenc}
\usepackage{endnotes}
\usepackage{epsfig}
\setlength{\parindent}{0pt}
\setlength{\parskip}{6pt}
\usepackage[spanish,mexico]{babel}
\usepackage{float}
\usepackage{xcolor}
\usepackage{dirtytalk}
\usepackage{pdftexcmds}
\usepackage{minted}
\usepackage{hyperref}

\begin{document}

\renewcommand\abstractname{Resumen}
\renewcommand\refname{Referencias}
\renewcommand{\notesname}{Notas}

\definecolor{plain}{RGB}{0,64,0}
\definecolor{forest}{RGB}{133,182,116}
\definecolor{sea}{RGB}{0,0,55}
\definecolor{coast}{RGB}{0,53,106}
\definecolor{mount}{RGB}{167,157,147}
\definecolor{mountain}{RGB}{216,223,226}

\newcommand\crule[3][black]{\textcolor{#1}{\rule{#2}{#3}}}

\title{Generación de elevaciones en múltiples núcleos}
\author{
\Large Saúl de Nova Caballero (A01165957)
\\
Tecnológico de Monterrey, Campus Estado de México
\\  
\Large \textit{A01165957@itesm.mx}}  

\maketitle

\begin{abstract}
Este artículo implementa generación de elevaciones de terreno. 
\end{abstract}

\section{Introducción}

\subsection{Antecedentes}

La generación de terreno es un tópico importante en videojuegos. Para agregar valor de rejuego (\textit{replayability} en inglés), muchos juegos deciden generar terreno nuevo cada vez que el usuario desea empezar de nuevo. A esto también se le conoce como generación procedural o generación por procedimientos.

Uno de los primeros ejemplos de generación procedural se pueden encontrar en el juego Rogue desarrollado en los 80s. Este juego se utilizó en muchos sistemas Unix.

\begin{figure}[H]
	\centering
		\epsfig{figure=rogue.eps,width=3in}
	\caption{Videojuego Rogue en un sistema moderno}
	\label{fig:rogue}	
\end{figure}

Un ejemplo de la complejidad que puedes llegar a alcanzar con generación procedural es el videojuego conocido como \textit{Slaves to Armok: God of Blood Chapter II: Dwarf Fortress} o en corto \textit{Dwarf Fortress}. Durante su generación procedural este juego genera:

\begin{enumerate}
	\item Terreno, Biomas, Ríos, Minerales
	\item Historia del mundo, incluyendo, creación de civilizaciones, guerras, invasiones, migraciones.
	\item Interacciones entre actores del mundo incluyendo su nacimiento y muerte.
	\item Personalidades a los actores
\end{enumerate}

Un ejemplo de la generación de terreno se puede encontrar en la figura \ref{fig:df}. Por limitaciones de tiempo en este artículo sólo se va a analizar la generación de elevaciones.

\begin{figure}[H]
	\centering
		\epsfig{figure=dwarffortress.eps,width=3in}
	\caption{Mapa generado en \textit{Dwarf Fortress}}
	\label{fig:df}	
\end{figure}

\subsection{Algoritmo de diamante-cuadro (\textit{Diamond-Square})}

Para generar la elevación del terreno se utilizó un algoritmo conocido como diamante-cuadro (\textit{Diamond-Square}). Este algoritmo fue propuesto por Alain Fournier, Don Fussell y Loren Carpenter en un artículo titulado \say{Representación computacional de modelos estocásticos} (\textit{Computer rendering of stochastic models}) en el SIGGRAPH de 1982. 

Este algoritmo genera un mapa de tamaño $n$ donde $n=2^k+1 \, con \, k \in Z^+$. Para la generación del mapa se utiliza el algoritmo mostrado en la figura \ref{fig:diamond_square}.

\begin{figure}[H]
	\centering
		\epsfig{figure=diamond_square.eps,width=6.4in}
	\caption{Algoritmo de diamante-cuadro}
	\label{fig:diamond_square}
\end{figure}

\pagebreak

\section{Desarrollo}

Para desarrollar este proyecto se decidieron utilizar las siguientes tecnologías:

\begin{itemize}
	\item \textit{Streams} paralelos en Java 8
	\item \textit{OpenMP} en C++
	\item \textit{Intrínsecos SIMD}	en C++
\end{itemize}

En el caso de \textit{OpenMP} e \textit{Intrínsecos SIMD} en C++ se realizó únicamente una versión secuencial para ambos casos. Esto debido a que realizar dos versiones secuenciales no tendría punto ya que serían iguales.

Todas las versiones imprimen sus resultados a archivos de mapas de bits  (.bmp) y archivos de texto (.txt). Se decidió utilizar estos dado que imprimir en este tipo de archivos es sencillo. Las figuras \ref{fig:map1} y \ref{fig:map2} muestran un ejemplo de este tipo de mapas. 

\begin{figure}[H]
	\centering
		\epsfig{figure=map.eps,width=4in}
	\caption{Mapa generado}
	\label{fig:map1}	
\end{figure}

\begin{figure}[H]
	\centering
		\epsfig{figure=map_parallel.eps,width=4in}
	\caption{Mapa generado}
	\label{fig:map2}	
\end{figure}

Los mapas se generaron con interpolación de 5 colores. Los colores se muestran en la tabla \ref{tab:colors}.

\begin{table}[H]
	\centering
	\begin{tabular}{|c|l|}
		\hline
			Color 							& Terreno		\\
		\hline
			\crule[plain]{0.2cm}{0.2cm} 		& Planicies 		\\
			\crule[coast]{0.2cm}{0.2cm} 		& Costa			\\
			\crule[sea]{0.2cm}{0.2cm} 		& Mar			\\
			\crule[mount]{0.2cm}{0.2cm} 		& Montes			\\
			\crule[mountain]{0.2cm}{0.2cm} 	& Montañas		\\
		\hline
	\end{tabular}
	\caption{Colores terreno}
	\label{tab:colors}
\end{table}

En el caso de los archivos de texto se utilizó la nomenclatura mostrada en la tabla \ref{tab:maptext}.

\begin{table}[H]
	\centering
	\begin{tabular}{|c|l|}
		\hline
			Caracter & Terreno		\\
		\hline
			$\wedge$& Planicies 		\\
			h		& Costa			\\
			p		& Mar			\\
			,		& Montes			\\
			. 		& Montañas		\\
		\hline
	\end{tabular}
	\caption{Colores terreno}
	\label{tab:maptext}
\end{table}

\pagebreak

Un ejemplo del archivo generado se puede encontrar a continuación:

\begin{verbatim}
p,,,pp,,,,.,,.,,,.............
p,,ppp,,,,,.,.................
p,,,,p,,,,....................
ppp,,,,,,,....................
ppp,pp,p,,.,..................
\end{verbatim}

\subsection{OpenMP en C++}

Para esta solución 

\subsubsection{Resultados}

\subsection{Intrínsecos SIMD en C++}

\subsubsection{Resultados}

\subsection{\textit{Streams} paralelos en Java}

La solución secuencial en Java utilizó el mismo algoritmo que se utilizó en C++. Mientras que la solución paralela consistió en implementar IntStreams en Java paralelizando los pasos de generación de diamantes y generación de cuadrados. 

\subsubsection{Resultados}

Los resultados de las corridas con \textit{streams} de Java son los siguientes.

\begin{table}[H]
	\centering
	\begin{tabular}{|c|c|c|c|}
		\hline
			& Tiempo secuencial & Tiempo paralelo & Speedup \\
		\hline
			Corrida 1 & 2.7778 & 1.1076 & 2.5078 \\
			Corrida 2 & 3.2013 & 0.8950 & 3.5769 \\
			Corrida 3 & 3.2982 & 1.3881 & 2.3759 \\
			Corrida 4 & 4.5891 & 0.9754 & 4.7043 \\
			Corrida 5 & 4.0893 & 1.0768 & 3.7975 \\
		\hline
			Promedio & 3.5911 & 1.0885 & 3.3924 \\
		\hline
	\end{tabular}
	\caption{Resultados streams}
	\label{tab:streams}
\end{table}

\begin{figure}[H]
	\centering
		\epsfig{figure=plot_java.eps,width=5in}
	\caption{Gráfica de \textit{streams} con Java}
	\label{fig:plot_stream}	
\end{figure}

Como podemos ver en Java el speedup es el único que es significativo. Esto puede ser debido a que Java no es tan rápido como C++ por lo que en este caso la mejora si es significativa.

\section{Conclusiones}

\section{Agradecimientos}

Tarn Adams por inspirar la creación de esta investigación.
 
\begin{thebibliography}{9}
    
\bibitem{crsm}  
    Departamento de letras del ITESM CEM. 
    \emph{Guía de presentación de trabajos escritos.} \\
    http://www.cem.itesm.mx/consulta/guia/ Accedido el 20 de octubre del 2015.
    
\bibitem{giybf}
    GIYBF. 
    \emph{Google is your best Friend!} \\
    http://www.giybf.com/ Accedido el 20 de octubre del 2015.
    
\bibitem{lamport}
    Lamport, L. 
    \emph{\LaTeX: A Document Preparation System, 2nd Edition.} 
    Addison-Wesley Professional, 1994.

\end{thebibliography}

\pagebreak

\appendix
\section{Código fuente}

El código fuente se encuentra disponible en la siguiente liga:\\
\url{https://github.com/sauldenova/parallel-terrain-generation}

\subsection{OpenMP e intrínsecos SIMD en C++}

\inputminted[linenos]{cpp}{../cpp/main.cxx}

\inputminted[linenos]{cpp}{../cpp/diamond_square.h}

\inputminted[linenos]{cpp}{../cpp/diamond_square.cxx}

\inputminted[linenos]{cpp}{../cpp/utilities/utilities.h}

\inputminted[linenos]{cpp}{../cpp/utilities/utilities.cxx}

\inputminted[linenos]{cpp}{../cpp/diamond_square_asm.h}

\inputminted[linenos]{nasm}{../cpp/diamond_square_asm.asm}

\subsection{\textit{Streams} paralelos en Java}

\inputminted[linenos]{java}{../java/src/com/terraingeneration/DiamondSquare.java}

\inputminted[linenos]{java}{../java/src/com/terraingeneration/TerrainGeneration.java}

\inputminted[linenos]{java}{../java/src/com/terraingeneration/utilities/Utilities.java}

\inputminted[linenos]{java}{../java/src/com/terraingeneration/utilities/Color.java}

\end{document}
